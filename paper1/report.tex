\documentclass[sigconf]{acmart}

\usepackage{hyperref}

\usepackage{endfloat}
\renewcommand{\efloatseparator}{\mbox{}} % no new page between figures

\usepackage{booktabs} % For formal tables

\settopmatter{printacmref=false} % Removes citation information below abstract
\renewcommand\footnotetextcopyrightpermission[1]{} % removes footnote with conference information in first column
\pagestyle{plain} % removes running headers

\begin{document}
\title{Big Data Applications in Electric Power Distribution}


\author{Swargam, Prashanth}
\affiliation{%
  \institution{Indiana University Bloomington}
  \streetaddress{107 S Indiana Ave}
  \city{Bloomington} 
  \state{Indiana} 
  \postcode{47408}
}
\email{pswargam@iu.edu}


% The default list of authors is too long for headers}
\renewcommand{\shortauthors}{B. Trovato et al.}


\begin{abstract}
Now-a-days, the process of storing the power measurements have changed. Conventional meters are replaced by the smart meters. New distribution management systems like SCADA and AMI are implemented to monitor power distribution. These smart meters record the readings and communicate the data to the server. However, these systems are designed to generate the readings very frequently i.e., 15 minutes to an hour. Upon that, smart meters are being deployed at every possible location to improve the accuracy of the data. This advancements in electric power distribution system results in enormous amounts of data which requires advance analytics to process, analyse and store data. This paper discusses about the implementation of Big Data technologies, challenges of implementing Big Data in Electric Power Distribution Systems. \cite{Munshi2017}
\end{abstract}

\keywords{Big Data, Power Distribution,Smart Power}

\maketitle

\section{Introduction}

Volume of data is increasing. According to forbes, it is said that, world’s data utilization will increase to 44 zettabytes from the current utilization of 4.4 zettabytes. To process this data, Big Data analytics will be useful. But, instantiating a big data architecture is not easy task. 

In electrical Power Distribution industry, data deluge is picking its pace. The data which was recorded for month, is now being noted for very small intervals. This quadruples  the amount of data that should be process. There is a lot of potential work to be put in for designing a good Big Data architecture to process and analyse this data. Most of the power generation units are developing their infrastructure to support these designs.

\subsection{Data Sources}

Smart meters which are placed at customer’s vicinity will record the consumption of a specific group of customers’. This data can be used to analyse the behaviour of customer for certain circumstances of weather and environment.

Distribution systems which manage the distribution of power, generate large amount of data related to voltages and currents at various levels of distribution. This data is very important in analysing the load level and demand for the distribution circle.

 Power measuring units at generation. This data is used to analyse the behaviour of generator and amount of power generation that will be required to supply enough power. This data will be used to decide the functioning of generators.
 
Old market data will be used to analyse the pricing and marketing strategies. These data is more focused on users and their behaviour. 


\subsection{4 v's in Big Data in Power Distribution System}
Volume: The data is periodically generated by many data sources like smart meters, machines and other appliances.
Variety: Each data source in electric power distribution system is explicit to each other. Each source has its own frequency of data generation and its own method of data generation. Thus, the data is heterogenous.
Velocity: is the speed at which the data is available for the end user.
Veracity: It deals with the correctness of the data. As all the data collected by sensors, meter tend to have various losses, correction algorithms should be defined to find the accurate data. Their might be chances for data transfer losses.

\bibliographystyle{ACM-Reference-Format}
\bibliography{report} 

\end{document}
