\documentclass[sigconf]{acmart}

\input{format/i523}

\begin{document}
\title{Big Data Applications in Aviation Industry}


\author{Swargam, Prashanth}

\affiliation{%
  \institution{Indiana University Bloomington}
  \streetaddress{107 S Indiana Ave}
  \city{Bloomington} 
  \state{Indiana} 
  \postcode{47408}
}
\email{pswargam@iu.edu}


% The default list of authors is too long for headers}
\renewcommand{\shortauthors}{B. Trovato et al.}


\begin{abstract}
Data generated by aviation industry is being increased enormously. The data generated by all the components of aviation industry can be analysed for reducing the operational costs, predict customer behaviour, analyse customer satisfaction. These applications of big data in aviation industry makes it a prominent player. Hence, collecting this data, storing and processing them for desired results can help the aviation industry in boosting their profits and improve customer satisfaction.Various applications of Big data, their challenges and models are discussed here.
\end{abstract}

\keywords{HID228, I523, Big Data Analytics, Aviation Industry}

\maketitle

\section{Introduction}

Big Data has transformed the businesses were being conducted. Every sector is integrated with data and generating huge amounts of data every day. All the companies are following data driven approach to crunch their competition. With the advent of concepts like Internet of things, the generation of data is increasing by many folds. This brings scope for a new business which handles the storage and analysis of data. 

The solutions offered by Big Data in many industrial sections have revolutionised the respective businesses. In aviation industry, where data as big as 20tb is generated from an aircraft flying for an hour\cite{Maire2017}, Big Data can offer influential solutions in terms of dealing with the massive data. This data ,if is processed in an efficient way would increase the customer satisfaction at a reduced running and operational costs which in turn increases the profits. 

\section{Applications}

\subsection{Baggage Handling}

All the customer check-in their bag and have a doubt if their bags are being transported with them. There are several cases where customers raise some complaints about their bag being missed are bag transported to another destination. Traditional barcode system was used to handle this task. As the number of airline users increased, this solution was not profitable for customers and airline operators. However, this is being replaced by the new technology which uses radio frequencies to track real-time location of the bag. Bags which are checked in at the kiosk are assigned with a microchip. These chips will send the data related to the location of the bag frequently.  The data generated by these chips is processed and stored. The processed data is available to the customers through mobile application or a web interface\cite{Miller2017}.
\subsection{Flight Safety}


All the flights have many sensors which generates a lot of data related to flight status and incidents. According to, a Being 737 generates nearly 20tb of data for one hour and an average cross international plane travelling for 6 hours generates 240 tb of data.  Most of these data is related to safety and status of various equipment on the flights. A lot of this data should be filtered and mined to generate a meaningful and usable data. Southwest Airlines partnered with NASA for crunching this data and generating a meaningful data. NASA uses machine learning algorithms to mine this data \cite{Smalley2012}. 


This collected data from the flight can be analysed to decide a desired value for variables like altitude, wind speed, thrust, weight of the aircraft are proposed to the pilot for increased fuel economy. This data can also be helpful in deciding the nature of services according to the nature of the location and fuel costs \cite{7889557}.

\subsection{Personalized promotion}

In the advent of smart devices, all the industries including airline industry have come closer to the customer. Variables which are considered as characteristics are studied from the customer data available through their interaction with customers. These details range from preferences to behaviour of the customer. This data is analysed to study the behaviour of the customer and improve his experience with the airline industry \cite{EXASTAX2017}. 

\subsection{Pricing strategies}

Pricing is an important strategy to generate profits. It is quite often to see a price bump of the airfare during the payment or checkout process. This is because of increase in demand for the journey. This demand data is analysed in the servers and a revised price is shown on the customers screen in less than minute. This calculations and analysis requires high computing power and efficient algorithms. EasyJet has uses  artificial intelligence to determine the price of seat based on deman d\cite{EXASTAX2017}. 


\section{Data Sources}

\subsection{In-Flight Data}

QAR Data: Quick Access Recorder records the statistics of the flight like speed, height, speed, altitude, at any instance during flight \cite{7015483}. This data is stored in servers and processed 

ACARS Data: Aircraft Communications Addressing and Reporting System is a online data transmission system which transmits data to ground through the aircraft’s satellite communication system\cite{Wikipedia}. ACARS records values of different parameters during a event. An event is an action performed by the aircraft. The sensors mounted on the flights’ brakes, wings, doors, send data to the ground staff using ACARS. Aircraft connection sensors and equipment monitoring system also uses this ACARS to transmit the data to ground.

\subsection{Data from mobile and web applications}

Now-a-days all the customer interactions with airline industry is through web. All the web applications and mobile applications which are developed for interacting with customer are smart enough to store the variables which are used to study the customer behaviour   \cite{EXASTAX2017}.  This portion of data sources generates the data at increasing rates due to the evolution of customer’s interaction with internet. 

\subsection{Historical Data}

Data available from the previous analytics and recordings constitutes a major portion. These are generally excel sheets or other forms of data stored in servers or files.  These can be used for predictive analysis of the flight.
\subsection{Other Sources}
Other sources like weather sensors, internet, analysis from third party vendors which help airline industry in scheduling and predicting flight delays .

\section{Cloud Storage in Aviation Industry}

According to\cite{Wikipedia2017}, Cloud computing is an IT process in which a specific application which belong to an organisation or individual is hosted on shared pool of servers. This data storage in these shared pools of servers can be provisioned on demand and can be resized according to the change in needs.

As aviation is industry produces enormous amounts of data, this requires high capacity computing servers to store and access them on demand. This makes the local storage and maintenance costly and time consuming. Airlines can outsource this activity by hosting their applications on cloud storage on any of the vendors either on public or private clouds\cite{6548579}. This model helps them in reducing costs and heavy IT infrastructure maintenance and allows more room for them to concentrate on their own business. 

In aviation industry, data is very rapid and requires faster storage and retrieval of data for efficient transactions. As these are shared pool of servers, these servers experience heavy data traffic. These servers are equipped with efficient load balancers. This ensures high availability and high speed of data access.

Customer Service can be increased without the increase in the IT infrastructure or IT workforce. As the infrastructure is outsourced, there will be minimal downtimes for activities like upgrades and maintenance This improves the customer experience\cite{Ferkoun2015}.

Aircraft maintenance involves changing and repairing various components of aircraft while keeping a complete track of all these changes and replacements. Cloud computing technologies offer good solutions and reduces the complexity of maintenance\cite{Ferkoun2015}. Architectures are developed to track these changes and providing this information and analytics to the appropriate people on their devices. 

Companies like Virgin Atlantic, Endeavour Air has implemented these solutions to enhance their Aircraft maintenance\cite{Vault2014}.


 

\section{Challenges in Implementing Big Data}
 
 

\subsection{Data Size}
With the advent of Internet of things, all the components of the aircraft are getting connected to the Internet. This enables communication between with the airline components and the ground staff. Every component detects its status and communicates it to the respective department. This results in generation of data and complexities in handling this huge data. According to \cite{Finnegan2013}, a Virgin Atlantic’s Boeing 787 generates approximately half terabyte of data per week.

\subsection{Data Format}

Data is produced in various formats by their respective sources. This results in high complexity in calculation and visualization. This complexity increases when, data needs to be transferred among various data sources or physical cloud databases\cite{6548579}. 
\subsection{Security}
In Aviation Industry, a large amounts of customer data are stored and processed for better analytics in marketing and promotion, this involves a huge risk. Almost a hundred and fifty parameters related to customer is taken from their interactions\cite{EXASTAX2017}, this data can be misused to locate customer by the attacking agents

As huge amounts of data related to aircraft and its maintenance is analysed and stored, this data can be vulnerable and can hamper aircraft security.

Advent of Cloud computing not only brings in advantages like lower IT maintenance cost, ease of other maintenance activities, but also brings in security issues alongside. Security measures higher than Industry standards must be used to protect data in shared pools of servers.

\section{Acknowledgements}

The author would like to thank Professor Gregor von Laszewski for providing all the help for this paper.

\section{Conclusion}

Big Data Analytics has provided impactful solutions in computing, storing and managing large amounts of data while lowering infrastructure costs and maintenance costs. This brief summary provides a overview of application of Big Data technologies in various aspects of Aviation Industry like Baggage handling, Aircraft safety, Customer Experience and Marketing. The discussion expands to various kinds to data sources for aviation industry and gives information about advantages of cloud storage. Though Big Data Analytics have provided prominent solutions in Aviation industry, it still has challenges related to security and protection of data. Research is being conducted in these areas to make more secure.


\bibliographystyle{ACM-Reference-Format}
\bibliography{report} 

\end{document}